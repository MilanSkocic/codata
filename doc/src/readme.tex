%-------------------------------------------------------------------------------
\section{Introduction}\label{introduction}
    \texttt{codata} is a Fortran library providing the fundamental physical constants
    according to 
    \href{https://www.nist.gov/programs-projects/codata-values-fundamental-physical-constants}{CODATA}.
    \citep{mohr2012, mohr2016, mohr2021, mohr2024, mohr2025}.
    A C API allows usage from C, or can be used as a basis for other wrappers. 
    Python wrapper allows easy usage from Python.

    The latest codata constants were released in 2022 by the \href{https://pml.nist.gov/cuu/Constants/}{NIST}. 
    All values for the codata constants are provided as double precision reals in 
    a derived type \texttt{codata\_constant\_type}. 
    The names are quite long and can be aliased with shorter names. 
    A module level interface to\_real is available for getting the constant value 
    or uncertainty of a constant.

    To use \texttt{codata} within your
    \href{https://github.com/fortran-lang/fpm}{fpm} project, add the
    following to your \texttt{fpm.toml} file:

    \begin{verbatim}
        [dependencies]
        codata = { git="https://github.com/MilanSkocic/codata.git" }
    \end{verbatim}

    \textbf{Notes}:

    \begin{itemize}
        \item The latest codata constants were integrated in the
          \href{https://github.com/fortran-lang/stdlib/releases/tag/v0.7.0}{stdlib}.
          The constants are implemented as derived type which carries the name,
          the value, the uncertainty and the unit. This library will be
          complementary to the constants defined in the stdlib by providing
          older values for the constants.
        \item If you only need sources for the codata constants that you can
          integrate directly in your sources you may be interested by
          https://github.com/vmagnin/fundamental\_constants.
    \end{itemize}

%-------------------------------------------------------------------------------
\section{Dependencies}\label{dependencies}
    \begin{verbatim}
        gfortran>=10
        fpm>=0.8
        stdlib>=0.5
        fypp>=3.0
    \end{verbatim}

%-------------------------------------------------------------------------------
\section{Installation}\label{installation}
    A Makefile is provided, which uses
    \href{https://fpm.fortran-lang.org}{fpm}, for building the library.

    \begin{itemize}
        \item On windows, \href{https://www.msys2.org}{msys2} needs to be installed.
          Add the msys2 binary (usually
          C:\textbackslash msys64\textbackslash usr\textbackslash bin) to the
          path in order to be able to use make.
        \item On Darwin, the \href{https://formulae.brew.sh/formula/gcc}{gcc}
          toolchain needs to be installed.
    \end{itemize}

    Build: the configuration file will set all the environment variables
    necessary for the compilation

    \begin{verbatim}
        chmod +x configure.sh
        ./configure.sh
        make
        make install
        make uninstall
    \end{verbatim}

%-------------------------------------------------------------------------------
\section{License}\label{license}
MIT
